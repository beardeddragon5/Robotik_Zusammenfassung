% !TEX root = ../Robotik.tex
\chapter{Prüfungsfragen (Schiedermeier)}
\paragraph{Wie viele Freiheitsgrade besitzt ein Flugzeug}
{2, 3, 4, 5, 6} (6)
\paragraph{Wie viele Freiheitsgrade besitzt ein Bodenfahrzeug}
2 Translatorische und 1 Rotatorischen
\paragraph{Was sind die Vorteile der Subsumption Architektur (multiple choice)}
\begin{itemize}
    \item Die einzelnen Verhalten sind unabhängig voneinander (true)
    \item Die einzelnen Verhalten sind einfach und leicht zu modellieren (true)
    \item Es existiert ein umfangreicher Gesamtplan (false)
	\item Auch mit wachsender Anzahl der Verhalten ist das Gesamtsystem leicht überschaubar und kombinierbar (false)
\end{itemize}
\paragraph{Für Schwarmrobotik wurden folgende Aussagen formuliert: Wählen Sie die richtigen Antworten aus}
\begin{itemize}
	\item Robertschwärme bestehen aus wenigen, einzelnen Agenten (false)
    \item Alle Tiere im Schwarm interagieren innerhalb eines fixen Radius (false)
    \item Schwarmintelligenz wird zur Lösung schwieriger Optimierungsprobleme eingesetzt (true)
    \item Die einzelnen Roboter im Schwarm besitzen ein einfaches Verhalten (true)
    \item Jeder einzelne Roboter im Schwarm benutzt auch globale Informationen (false)
    \item Jeder einzelne Roboter im Schwarm trifft Entscheidungen aufgrund lokaler Informationen (true)
    \item Das Kollektiv ist intelligenter als das Individuum (true)
\end{itemize}
\paragraph{Topologische Karten sind gut zur Pfadplanung geeignet, weil viele geeignete Algorithmen für Graphen existieren? Wahr oder falsch?}
Wahr bsp. Dijkstra, A*
\paragraph{Als Landmarken eignet sich alles, was von unterschiedlichen Positionen aus gut sichtbar ist}
Wahr
\paragraph{Was sind künstliche Landmarken (multiple choice)}
\begin{itemize}
	\item Fest verbaute Schränke (false)
    \item farbige Elemente wie z.B. gelbe Tore beim RoboCup (true)
    \item Ampeln (true)
    \item Wände (false)
    \item Peilsender (true)
    \item Türen (false)
    \item Barcodestreifen (true)
\end{itemize}
\paragraph{Zur exakten Positionsbestimmung benötigt der Roboter eine bestimmte Anzahl Landmarken. Wie viele sind es?}
Landmarken >= 2
\paragraph{Wie viele Stützfüße sind notwendig, damit ein Stützpolygon für den Massenschwerpunkt aufgebaut werden kann?}
2s
\paragraph{Warum sind beim Dijkstra Algorithmus keine negativen Kantengewichte erlaubt?}
\paragraph{Nach welchen Regeln bewegen sich Agenten im Schwarm?}
\paragraph{In welcher Reihenfolge werden beim Dijkstra Algorithmus die Knoten aus der Warteschlange entnommen?}
\paragraph{Beschreiben Sie mit eigenen Worten das Vorgehen beim Sichtgraph Algorithmus. Welche Nachteile bringt diese Methode mit?}
\paragraph{Welche Nachteile haben Voronoi Diagramme gegenüber dem Sichtgraph Algorithmus? (multiple choice)}
\begin{itemize}
	\item Die Planung kann ineffiziente Wege erzeugen. (true)
    \item Die Wege führen wie beim Sichtgraph Algorithmus nahe an den Hindernissen.(false)
    \item Bei Voronoi-Diagrammen werden große Freiflächen nur durch wenige Ecken und Verbindungen aufgespannt.(true)
    \item Der Graph enthält wie beim Sichtgraph Algorithmus viele nutzlose Kanten (false)
\end{itemize}
\paragraph{Warum finden Bug Algorithmen selten den optimalen Weg zum Ziel?}
\paragraph{Laufroboter haben Vorteile gegenüber mobilen Robotern auf Rädern}
\begin{itemize}
	\item Beine kommen mit (einzelnen oder mehreren) Trittstellen aus. (\textbf{einzelnen})
    \item Beine können über (Geroll und Löcher oder größere Felsbrocken) steigen. (\textbf{Geroll und Löcher})
    \item Beine kompensieren Unebenheiten durch Anpassung der (Schrittanzahl oder Schritthöhe und -länge). (\textbf{Schritthöhe und-länge})
\end{itemize}
