% !TEX root = ../Robotik.tex
\chapter{Fortbewegung, Lokalisierungsalgorithmen}
\section{Relative Lokalisierung}
\subsection{Dead Reckoning}
\begin{itemize}
	\item \textbf{Koppelnavigation oder Dead Reckoning} ursprünglich in der
		Nautik verwendet
	\item Mathematisches Verfahren - \textbf{Vorwärtskinematik} - zur
		Positionsbestimmung
	\item Ausgehend von einer Startposition ist es dem Navigator möglich, seine
		\textbf{aktuelle Position zu berechnen} aufgrund der
		\textbf{zurückliegenden bekannten Kurs- und Geschwindigkeitswerte}
\end{itemize}
\begin{figure}[H]
	\begin{center}
		\includegraphics[scale=0.5]{Resources/PDF/deadReckoning.pdf}
		\caption{}
		\label{fig:PDF/deadReckoning.pdf}
	\end{center}
\end{figure}
\begin{itemize}
	\item A sei ein gegebener Ausgangspunkt
	\item Radius wird angegeben der zur Abweichung proportional ist, bsp. hier
		0,5m.
	\item Der Radius spiegelt die mit der Zeit kumulierte Ungenauigkeit wieder
	\item Eine mögliche Roboterposition ist dann innerhalb des Sektors gegeben
		der durch die Linien eingegrenzt wird
\end{itemize}

\paragraph{Vorteile}
\begin{itemize}
	\item Einfache Implementierung
	\item Leichte Interpretation der Daten
	\item Unkomplizierte Bedienung
	\item Passable Kurzstreckengenauigkeit
\end{itemize}

\paragraph{Nachteile}
\begin{itemize}
	\item Startposition muss bekannt sein
	\item Genauigkeit nimmt mit zunehmender Länge der befahrenen Strecke
		drastisch ab
\end{itemize}

\subsection{Odometrie}
Odometrie ist die Wissenschaft der Positionsbestimmung eines Fahrzeugs durch
die Beobachtung seiner Räder.

\paragraph{Grundlegendes verfahren}
\begin{itemize}
	\item Sensoren an Rädern messen Drehbewegung
	\item \textbf{Relative Positionsbestimmung:} Die \textbf{Bestimmung der
		Position} erfolgt ausgehend von einer bekannten Position durch Berechnung
		des zurückgelegten Weges und anhand von Daten über den Roboter selbst.
	\item Es wird Inkrementalgebern die Anzahl $n$ der Radumdrehungen zwischen
		zwei Messpunkten gezählt. Aus dem bekannten Radumfang wird die
		wegdifferenz berechnet mit:
		\subitem $\Delta = \Pi \times d \times n $
	\item \textbf{Ausrichtung} kann durch differentiale Odometrie erfolgen: es
		werden z.B. die unterschiedlichen Entfernungen gemessen, die die linken
		und rechten Räder zurückgelegt haben.
\end{itemize}

\paragraph{Vorteile}
\begin{itemize}
	\item kostengünstig
	\item hohe Abtastraten
	\item passable Kurzzeitgenauigkeit
\end{itemize}

\paragraph{Fehlerquellen}
\begin{itemize}
	\item Fehlerhafte Messung des Raddurchmessers
	\item Raddurchmesser nicht gleich, Unrundheit des Rades
\end{itemize}

\paragraph{Fehlerberücksichtigung}
\begin{itemize}
	\item Die \textbf{Fehler} fließen in die Positionsdifferenz ein, werden zur
		letzten bekannten Position hinzuaddiert und \textbf{summieren sich mit
		jedem Messschritt}
	\item Fehlerellipse wächst mit zurückgelegtem Weg
	\item Odometrie als alleiniges Verfahren nur für kurze Strecken geeignet
	\item Fehler lassen sich bei geringen Geschwindigkeiten und geringer
		Beschleunigung reduziern
\end{itemize}

\subsection{2D-Scanmatching}
\begin{itemize}
	\item Ausgangslage sind zwei Scans, ein Scan M (\textbf{Modell}) und ein
		zweiter Scan D (\textbf{Daten})
	\item Es wird eine Transformation des einen Scans berechnet und zwar so,
		dass beide optimal überlagert werden
	\item Die Transformationen bestehen nur aus einer Rotation und einer
		Translation
	\item Die Überlagerung ist optimal, wenn Punkte, die in der realen Szene
		nahe beieinander liegen, auch in den registrierten Messdaten nahe
		beieinander liegen.
	\item \textbf{Ziel}: Fehlerfunktion minimieren $\Rightarrow$ Abstände der
		Punkte des einen Scans zu ihren korrespondierenden Punkten des zweiten
		Scans
	\item Die Transformation des zweiten Scans entspricht dann der Bewegung des
		Roboters zwischen der Aufnahme der Daten; durch sukzessiven Vergleich kann
		damit die Bewegung des Roboters nachverfolgt werden
\end{itemize}

\paragraph{Iteratives Vorgehen}
\begin{figure}[H]
	\begin{center}
		\includegraphics[scale=0.3]{Resources/PNG/2DScan.PNG}
		\caption{}
		\label{fig:PNG/2DScan.PNG}
	\end{center}
\end{figure}
\begin{itemize}
	\item \textbf{Annahme:} die korrespondierenden Punkte sind bekannt
		$\Rightarrow$ eine \textbf{Transformation} kann berechnet werden, die
		diese Mengen aufeinander abbildet
	\item \textbf{Beispiel} zwei Scans mit einer Poseänderung des zweiten um
		$(4cm, 1cm, 10 \deg)^T$ beide Scans sehen dieselben Raumpunkte $p_1$ und 
		$p_2$
	\item Obige Annahme i.d.r. nicht erfüllt $\Rightarrow$ \textbf{nicht
		eindeutig zu bestimmen, welche Punkte zwischen den beiden Scans
		korresponideren}
	\item \textbf{Lösung}: \textbf{iteratives Vorgehen}, bei dem zunächst eine
		\textbf{Schätzung} der Punktepaarung stattfinden und die Pose des zweiten
		Scans unter dieser Paarung optimiert wird.
	\item Iterativ werden mit dem transformierten Scan neue Punktepaare
		berechnet, bis ein Abbruchkriterium erfüllt ist, d.h. bis sich die
		Transformation zwischen zwei Schritten nich mehr signifikant ändert
\end{itemize}

\paragraph{Transformationsberechnung}
\begin{itemize}
	\item \textbf{Gesucht}: Mögliche Menge von Translationen und Rotationen,
		unter denen ein korrektes Matching möglich ist.
	\item $(t_x, t_z, \theta)^T$, die eine Translation um $t_x$ in x-Richtung
		und $t_z$ entlang der Z-Achse durchführt, sowie eine Rotation um den
		Winkel $\theta$
	\item Der Scan M besteht aus einer Menge von Punkten $(m_i)_{i=1,2 \dots N}$
	\item Der Scan D besteht aus einer Menge von Punkten $(D_i)_{i=1,2 \dots N}$
\end{itemize}

\paragraph{Minimum der Funktion}
$$
	E(\theta, t)=\sum_{i=1}^N \left\| p_i - \left( \boldsymbol R_\theta
		\boldsymbol p'_i + \boldsymbol t \right) \right\| ^2
$$

\paragraph{Transformation zur mimierung der Fehlerfunktion E} \hfill \\
Folgende Transformation mit ggb. Parametern minimiert die Fehlerfunktion:
$$
	\theta=\arctan \left(\frac{S_{z x^{\prime}}-S_{x z^{\prime}}}{S_{x
		x^{\prime}}+S_{z z^{\prime}}}\right)
$$
Hierbei ist
\begin{align*}
	t_x &= c_x - \left( c'_x \cos\theta - c'_z \sin\theta \right) \\
	t_z &= c_z - \left( c'_x \sin\theta + c'_z \cos\theta \right) \\
\end{align*}
mit den Parametern:
\begin{align*}
	c_x &= \frac1N \sum_i p_{x,i} \qquad & S_{xx'} = \sum_i (p_{x,i} - c_x)
		(p'_{x,i} - c'_x) \\
	c_z &= \frac1N \sum_i p_{z,i} \qquad & S_{xz'} = \sum_i (p_{x,i} - c_x)
		(p'_{z,i} - c'_z) \\
	c'_x &= \frac1N \sum_i p'_{x,i} \qquad & S_{zx'} = \sum_i (p_{z,i} - c_z)
		(p'_{x,i} - c'_x) \\
	c'_z &= \frac1N \sum_i p'_{z,i} \qquad & S_{zz'} = \sum_i (p_{z,i} - c_z)
		(p'_{z,i} - c'_z)
\end{align*}

%TODO Look at BSP 26 / 28
