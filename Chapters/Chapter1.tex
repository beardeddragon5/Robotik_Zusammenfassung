% !TEX root = ../Robotik.tex
\chapter{Geschichte}
\section{Industrieroboter}
Nach Definition der VDI-Richtlinie 2860 sind Industrieroboter universell
einsetzbare Bewegungsautomaten mit meherern Achsen, deren Bewegungen
hinsichtlich Bewegungsfolge und Wegen bzw. Winkel frei programmierbar und
sensorgeführt sind.

Zeichen sich aus durch:
\begin{itemize}
	\item Schnelligkeit
	\item Genauigkeit
	\item Robustheit
	\item Traglast
\end{itemize}

Einsatzgebiete sind unter anderen \textbf{Schweißen, Kleben, Schneiden,
Lackieren.}

Zunehmend \textbf{kollaborative} Roboter, Cobots die mit Menschen \textbf{ohne
Schutzeinrichtungen} im Produktionsprozess interagieren und diese
\textbf{Wahrnehmen um Verletzungen zu vermeiden}.

\section{Serviceroboter}
Ein \textbf{Serviceroboter} ist eine \textbf{frei programmierbare
Bewegungseinrichtung}, die \textbf{teil- oder vollautomatisch} Dienstleistungen
verrichten.

\textbf{Dienstleistungen} sind dabei Tätigkeiten, die \textbf{nicht der
direkten industriellen Erzeugung} von Sach\-gütern, sonder Verrichtung von
\textbf{Leistungen für Menschen und Einrichtungen} dienen. Sie werden in
zwei Klassen eingeteilt:

\begin{description}
	\item[Professionel] Einsatzbereiche sind beispielsweise:
		\begin{itemize}
			\item Rettung
			\item Landwirtschaft
			\item Medizin
		\end{itemize}
	\item[Privat] Einsatzbereiche sind beispielsweise:
		\begin{itemize}
			\item Rasenmäher
			\item Pfleger
		\end{itemize}
\end{description}
